
% Default to the notebook output style

    


% Inherit from the specified cell style.




    
\documentclass[11pt]{article}

    
    
    \usepackage[T1]{fontenc}
    % Nicer default font (+ math font) than Computer Modern for most use cases
    \usepackage{mathpazo}

    % Basic figure setup, for now with no caption control since it's done
    % automatically by Pandoc (which extracts ![](path) syntax from Markdown).
    \usepackage{graphicx}
    % We will generate all images so they have a width \maxwidth. This means
    % that they will get their normal width if they fit onto the page, but
    % are scaled down if they would overflow the margins.
    \makeatletter
    \def\maxwidth{\ifdim\Gin@nat@width>\linewidth\linewidth
    \else\Gin@nat@width\fi}
    \makeatother
    \let\Oldincludegraphics\includegraphics
    % Set max figure width to be 80% of text width, for now hardcoded.
    \renewcommand{\includegraphics}[1]{\Oldincludegraphics[width=.8\maxwidth]{#1}}
    % Ensure that by default, figures have no caption (until we provide a
    % proper Figure object with a Caption API and a way to capture that
    % in the conversion process - todo).
    \usepackage{caption}
    \DeclareCaptionLabelFormat{nolabel}{}
    \captionsetup{labelformat=nolabel}

    \usepackage{adjustbox} % Used to constrain images to a maximum size 
    \usepackage{xcolor} % Allow colors to be defined
    \usepackage{enumerate} % Needed for markdown enumerations to work
    \usepackage{geometry} % Used to adjust the document margins
    \usepackage{amsmath} % Equations
    \usepackage{amssymb} % Equations
    \usepackage{textcomp} % defines textquotesingle
    % Hack from http://tex.stackexchange.com/a/47451/13684:
    \AtBeginDocument{%
        \def\PYZsq{\textquotesingle}% Upright quotes in Pygmentized code
    }
    \usepackage{upquote} % Upright quotes for verbatim code
    \usepackage{eurosym} % defines \euro
    \usepackage[mathletters]{ucs} % Extended unicode (utf-8) support
    \usepackage[utf8x]{inputenc} % Allow utf-8 characters in the tex document
    \usepackage{fancyvrb} % verbatim replacement that allows latex
    \usepackage{grffile} % extends the file name processing of package graphics 
                         % to support a larger range 
    % The hyperref package gives us a pdf with properly built
    % internal navigation ('pdf bookmarks' for the table of contents,
    % internal cross-reference links, web links for URLs, etc.)
    \usepackage{hyperref}
    \usepackage{longtable} % longtable support required by pandoc >1.10
    \usepackage{booktabs}  % table support for pandoc > 1.12.2
    \usepackage[inline]{enumitem} % IRkernel/repr support (it uses the enumerate* environment)
    \usepackage[normalem]{ulem} % ulem is needed to support strikethroughs (\sout)
                                % normalem makes italics be italics, not underlines
    

    
    
    % Colors for the hyperref package
    \definecolor{urlcolor}{rgb}{0,.145,.698}
    \definecolor{linkcolor}{rgb}{.71,0.21,0.01}
    \definecolor{citecolor}{rgb}{.12,.54,.11}

    % ANSI colors
    \definecolor{ansi-black}{HTML}{3E424D}
    \definecolor{ansi-black-intense}{HTML}{282C36}
    \definecolor{ansi-red}{HTML}{E75C58}
    \definecolor{ansi-red-intense}{HTML}{B22B31}
    \definecolor{ansi-green}{HTML}{00A250}
    \definecolor{ansi-green-intense}{HTML}{007427}
    \definecolor{ansi-yellow}{HTML}{DDB62B}
    \definecolor{ansi-yellow-intense}{HTML}{B27D12}
    \definecolor{ansi-blue}{HTML}{208FFB}
    \definecolor{ansi-blue-intense}{HTML}{0065CA}
    \definecolor{ansi-magenta}{HTML}{D160C4}
    \definecolor{ansi-magenta-intense}{HTML}{A03196}
    \definecolor{ansi-cyan}{HTML}{60C6C8}
    \definecolor{ansi-cyan-intense}{HTML}{258F8F}
    \definecolor{ansi-white}{HTML}{C5C1B4}
    \definecolor{ansi-white-intense}{HTML}{A1A6B2}

    % commands and environments needed by pandoc snippets
    % extracted from the output of `pandoc -s`
    \providecommand{\tightlist}{%
      \setlength{\itemsep}{0pt}\setlength{\parskip}{0pt}}
    \DefineVerbatimEnvironment{Highlighting}{Verbatim}{commandchars=\\\{\}}
    % Add ',fontsize=\small' for more characters per line
    \newenvironment{Shaded}{}{}
    \newcommand{\KeywordTok}[1]{\textcolor[rgb]{0.00,0.44,0.13}{\textbf{{#1}}}}
    \newcommand{\DataTypeTok}[1]{\textcolor[rgb]{0.56,0.13,0.00}{{#1}}}
    \newcommand{\DecValTok}[1]{\textcolor[rgb]{0.25,0.63,0.44}{{#1}}}
    \newcommand{\BaseNTok}[1]{\textcolor[rgb]{0.25,0.63,0.44}{{#1}}}
    \newcommand{\FloatTok}[1]{\textcolor[rgb]{0.25,0.63,0.44}{{#1}}}
    \newcommand{\CharTok}[1]{\textcolor[rgb]{0.25,0.44,0.63}{{#1}}}
    \newcommand{\StringTok}[1]{\textcolor[rgb]{0.25,0.44,0.63}{{#1}}}
    \newcommand{\CommentTok}[1]{\textcolor[rgb]{0.38,0.63,0.69}{\textit{{#1}}}}
    \newcommand{\OtherTok}[1]{\textcolor[rgb]{0.00,0.44,0.13}{{#1}}}
    \newcommand{\AlertTok}[1]{\textcolor[rgb]{1.00,0.00,0.00}{\textbf{{#1}}}}
    \newcommand{\FunctionTok}[1]{\textcolor[rgb]{0.02,0.16,0.49}{{#1}}}
    \newcommand{\RegionMarkerTok}[1]{{#1}}
    \newcommand{\ErrorTok}[1]{\textcolor[rgb]{1.00,0.00,0.00}{\textbf{{#1}}}}
    \newcommand{\NormalTok}[1]{{#1}}
    
    % Additional commands for more recent versions of Pandoc
    \newcommand{\ConstantTok}[1]{\textcolor[rgb]{0.53,0.00,0.00}{{#1}}}
    \newcommand{\SpecialCharTok}[1]{\textcolor[rgb]{0.25,0.44,0.63}{{#1}}}
    \newcommand{\VerbatimStringTok}[1]{\textcolor[rgb]{0.25,0.44,0.63}{{#1}}}
    \newcommand{\SpecialStringTok}[1]{\textcolor[rgb]{0.73,0.40,0.53}{{#1}}}
    \newcommand{\ImportTok}[1]{{#1}}
    \newcommand{\DocumentationTok}[1]{\textcolor[rgb]{0.73,0.13,0.13}{\textit{{#1}}}}
    \newcommand{\AnnotationTok}[1]{\textcolor[rgb]{0.38,0.63,0.69}{\textbf{\textit{{#1}}}}}
    \newcommand{\CommentVarTok}[1]{\textcolor[rgb]{0.38,0.63,0.69}{\textbf{\textit{{#1}}}}}
    \newcommand{\VariableTok}[1]{\textcolor[rgb]{0.10,0.09,0.49}{{#1}}}
    \newcommand{\ControlFlowTok}[1]{\textcolor[rgb]{0.00,0.44,0.13}{\textbf{{#1}}}}
    \newcommand{\OperatorTok}[1]{\textcolor[rgb]{0.40,0.40,0.40}{{#1}}}
    \newcommand{\BuiltInTok}[1]{{#1}}
    \newcommand{\ExtensionTok}[1]{{#1}}
    \newcommand{\PreprocessorTok}[1]{\textcolor[rgb]{0.74,0.48,0.00}{{#1}}}
    \newcommand{\AttributeTok}[1]{\textcolor[rgb]{0.49,0.56,0.16}{{#1}}}
    \newcommand{\InformationTok}[1]{\textcolor[rgb]{0.38,0.63,0.69}{\textbf{\textit{{#1}}}}}
    \newcommand{\WarningTok}[1]{\textcolor[rgb]{0.38,0.63,0.69}{\textbf{\textit{{#1}}}}}
    
    
    % Define a nice break command that doesn't care if a line doesn't already
    % exist.
    \def\br{\hspace*{\fill} \\* }
    % Math Jax compatability definitions
    \def\gt{>}
    \def\lt{<}
    % Document parameters
    \title{Conding with Python}
    
    
    

    % Pygments definitions
    
\makeatletter
\def\PY@reset{\let\PY@it=\relax \let\PY@bf=\relax%
    \let\PY@ul=\relax \let\PY@tc=\relax%
    \let\PY@bc=\relax \let\PY@ff=\relax}
\def\PY@tok#1{\csname PY@tok@#1\endcsname}
\def\PY@toks#1+{\ifx\relax#1\empty\else%
    \PY@tok{#1}\expandafter\PY@toks\fi}
\def\PY@do#1{\PY@bc{\PY@tc{\PY@ul{%
    \PY@it{\PY@bf{\PY@ff{#1}}}}}}}
\def\PY#1#2{\PY@reset\PY@toks#1+\relax+\PY@do{#2}}

\expandafter\def\csname PY@tok@w\endcsname{\def\PY@tc##1{\textcolor[rgb]{0.73,0.73,0.73}{##1}}}
\expandafter\def\csname PY@tok@c\endcsname{\let\PY@it=\textit\def\PY@tc##1{\textcolor[rgb]{0.25,0.50,0.50}{##1}}}
\expandafter\def\csname PY@tok@cp\endcsname{\def\PY@tc##1{\textcolor[rgb]{0.74,0.48,0.00}{##1}}}
\expandafter\def\csname PY@tok@k\endcsname{\let\PY@bf=\textbf\def\PY@tc##1{\textcolor[rgb]{0.00,0.50,0.00}{##1}}}
\expandafter\def\csname PY@tok@kp\endcsname{\def\PY@tc##1{\textcolor[rgb]{0.00,0.50,0.00}{##1}}}
\expandafter\def\csname PY@tok@kt\endcsname{\def\PY@tc##1{\textcolor[rgb]{0.69,0.00,0.25}{##1}}}
\expandafter\def\csname PY@tok@o\endcsname{\def\PY@tc##1{\textcolor[rgb]{0.40,0.40,0.40}{##1}}}
\expandafter\def\csname PY@tok@ow\endcsname{\let\PY@bf=\textbf\def\PY@tc##1{\textcolor[rgb]{0.67,0.13,1.00}{##1}}}
\expandafter\def\csname PY@tok@nb\endcsname{\def\PY@tc##1{\textcolor[rgb]{0.00,0.50,0.00}{##1}}}
\expandafter\def\csname PY@tok@nf\endcsname{\def\PY@tc##1{\textcolor[rgb]{0.00,0.00,1.00}{##1}}}
\expandafter\def\csname PY@tok@nc\endcsname{\let\PY@bf=\textbf\def\PY@tc##1{\textcolor[rgb]{0.00,0.00,1.00}{##1}}}
\expandafter\def\csname PY@tok@nn\endcsname{\let\PY@bf=\textbf\def\PY@tc##1{\textcolor[rgb]{0.00,0.00,1.00}{##1}}}
\expandafter\def\csname PY@tok@ne\endcsname{\let\PY@bf=\textbf\def\PY@tc##1{\textcolor[rgb]{0.82,0.25,0.23}{##1}}}
\expandafter\def\csname PY@tok@nv\endcsname{\def\PY@tc##1{\textcolor[rgb]{0.10,0.09,0.49}{##1}}}
\expandafter\def\csname PY@tok@no\endcsname{\def\PY@tc##1{\textcolor[rgb]{0.53,0.00,0.00}{##1}}}
\expandafter\def\csname PY@tok@nl\endcsname{\def\PY@tc##1{\textcolor[rgb]{0.63,0.63,0.00}{##1}}}
\expandafter\def\csname PY@tok@ni\endcsname{\let\PY@bf=\textbf\def\PY@tc##1{\textcolor[rgb]{0.60,0.60,0.60}{##1}}}
\expandafter\def\csname PY@tok@na\endcsname{\def\PY@tc##1{\textcolor[rgb]{0.49,0.56,0.16}{##1}}}
\expandafter\def\csname PY@tok@nt\endcsname{\let\PY@bf=\textbf\def\PY@tc##1{\textcolor[rgb]{0.00,0.50,0.00}{##1}}}
\expandafter\def\csname PY@tok@nd\endcsname{\def\PY@tc##1{\textcolor[rgb]{0.67,0.13,1.00}{##1}}}
\expandafter\def\csname PY@tok@s\endcsname{\def\PY@tc##1{\textcolor[rgb]{0.73,0.13,0.13}{##1}}}
\expandafter\def\csname PY@tok@sd\endcsname{\let\PY@it=\textit\def\PY@tc##1{\textcolor[rgb]{0.73,0.13,0.13}{##1}}}
\expandafter\def\csname PY@tok@si\endcsname{\let\PY@bf=\textbf\def\PY@tc##1{\textcolor[rgb]{0.73,0.40,0.53}{##1}}}
\expandafter\def\csname PY@tok@se\endcsname{\let\PY@bf=\textbf\def\PY@tc##1{\textcolor[rgb]{0.73,0.40,0.13}{##1}}}
\expandafter\def\csname PY@tok@sr\endcsname{\def\PY@tc##1{\textcolor[rgb]{0.73,0.40,0.53}{##1}}}
\expandafter\def\csname PY@tok@ss\endcsname{\def\PY@tc##1{\textcolor[rgb]{0.10,0.09,0.49}{##1}}}
\expandafter\def\csname PY@tok@sx\endcsname{\def\PY@tc##1{\textcolor[rgb]{0.00,0.50,0.00}{##1}}}
\expandafter\def\csname PY@tok@m\endcsname{\def\PY@tc##1{\textcolor[rgb]{0.40,0.40,0.40}{##1}}}
\expandafter\def\csname PY@tok@gh\endcsname{\let\PY@bf=\textbf\def\PY@tc##1{\textcolor[rgb]{0.00,0.00,0.50}{##1}}}
\expandafter\def\csname PY@tok@gu\endcsname{\let\PY@bf=\textbf\def\PY@tc##1{\textcolor[rgb]{0.50,0.00,0.50}{##1}}}
\expandafter\def\csname PY@tok@gd\endcsname{\def\PY@tc##1{\textcolor[rgb]{0.63,0.00,0.00}{##1}}}
\expandafter\def\csname PY@tok@gi\endcsname{\def\PY@tc##1{\textcolor[rgb]{0.00,0.63,0.00}{##1}}}
\expandafter\def\csname PY@tok@gr\endcsname{\def\PY@tc##1{\textcolor[rgb]{1.00,0.00,0.00}{##1}}}
\expandafter\def\csname PY@tok@ge\endcsname{\let\PY@it=\textit}
\expandafter\def\csname PY@tok@gs\endcsname{\let\PY@bf=\textbf}
\expandafter\def\csname PY@tok@gp\endcsname{\let\PY@bf=\textbf\def\PY@tc##1{\textcolor[rgb]{0.00,0.00,0.50}{##1}}}
\expandafter\def\csname PY@tok@go\endcsname{\def\PY@tc##1{\textcolor[rgb]{0.53,0.53,0.53}{##1}}}
\expandafter\def\csname PY@tok@gt\endcsname{\def\PY@tc##1{\textcolor[rgb]{0.00,0.27,0.87}{##1}}}
\expandafter\def\csname PY@tok@err\endcsname{\def\PY@bc##1{\setlength{\fboxsep}{0pt}\fcolorbox[rgb]{1.00,0.00,0.00}{1,1,1}{\strut ##1}}}
\expandafter\def\csname PY@tok@kc\endcsname{\let\PY@bf=\textbf\def\PY@tc##1{\textcolor[rgb]{0.00,0.50,0.00}{##1}}}
\expandafter\def\csname PY@tok@kd\endcsname{\let\PY@bf=\textbf\def\PY@tc##1{\textcolor[rgb]{0.00,0.50,0.00}{##1}}}
\expandafter\def\csname PY@tok@kn\endcsname{\let\PY@bf=\textbf\def\PY@tc##1{\textcolor[rgb]{0.00,0.50,0.00}{##1}}}
\expandafter\def\csname PY@tok@kr\endcsname{\let\PY@bf=\textbf\def\PY@tc##1{\textcolor[rgb]{0.00,0.50,0.00}{##1}}}
\expandafter\def\csname PY@tok@bp\endcsname{\def\PY@tc##1{\textcolor[rgb]{0.00,0.50,0.00}{##1}}}
\expandafter\def\csname PY@tok@fm\endcsname{\def\PY@tc##1{\textcolor[rgb]{0.00,0.00,1.00}{##1}}}
\expandafter\def\csname PY@tok@vc\endcsname{\def\PY@tc##1{\textcolor[rgb]{0.10,0.09,0.49}{##1}}}
\expandafter\def\csname PY@tok@vg\endcsname{\def\PY@tc##1{\textcolor[rgb]{0.10,0.09,0.49}{##1}}}
\expandafter\def\csname PY@tok@vi\endcsname{\def\PY@tc##1{\textcolor[rgb]{0.10,0.09,0.49}{##1}}}
\expandafter\def\csname PY@tok@vm\endcsname{\def\PY@tc##1{\textcolor[rgb]{0.10,0.09,0.49}{##1}}}
\expandafter\def\csname PY@tok@sa\endcsname{\def\PY@tc##1{\textcolor[rgb]{0.73,0.13,0.13}{##1}}}
\expandafter\def\csname PY@tok@sb\endcsname{\def\PY@tc##1{\textcolor[rgb]{0.73,0.13,0.13}{##1}}}
\expandafter\def\csname PY@tok@sc\endcsname{\def\PY@tc##1{\textcolor[rgb]{0.73,0.13,0.13}{##1}}}
\expandafter\def\csname PY@tok@dl\endcsname{\def\PY@tc##1{\textcolor[rgb]{0.73,0.13,0.13}{##1}}}
\expandafter\def\csname PY@tok@s2\endcsname{\def\PY@tc##1{\textcolor[rgb]{0.73,0.13,0.13}{##1}}}
\expandafter\def\csname PY@tok@sh\endcsname{\def\PY@tc##1{\textcolor[rgb]{0.73,0.13,0.13}{##1}}}
\expandafter\def\csname PY@tok@s1\endcsname{\def\PY@tc##1{\textcolor[rgb]{0.73,0.13,0.13}{##1}}}
\expandafter\def\csname PY@tok@mb\endcsname{\def\PY@tc##1{\textcolor[rgb]{0.40,0.40,0.40}{##1}}}
\expandafter\def\csname PY@tok@mf\endcsname{\def\PY@tc##1{\textcolor[rgb]{0.40,0.40,0.40}{##1}}}
\expandafter\def\csname PY@tok@mh\endcsname{\def\PY@tc##1{\textcolor[rgb]{0.40,0.40,0.40}{##1}}}
\expandafter\def\csname PY@tok@mi\endcsname{\def\PY@tc##1{\textcolor[rgb]{0.40,0.40,0.40}{##1}}}
\expandafter\def\csname PY@tok@il\endcsname{\def\PY@tc##1{\textcolor[rgb]{0.40,0.40,0.40}{##1}}}
\expandafter\def\csname PY@tok@mo\endcsname{\def\PY@tc##1{\textcolor[rgb]{0.40,0.40,0.40}{##1}}}
\expandafter\def\csname PY@tok@ch\endcsname{\let\PY@it=\textit\def\PY@tc##1{\textcolor[rgb]{0.25,0.50,0.50}{##1}}}
\expandafter\def\csname PY@tok@cm\endcsname{\let\PY@it=\textit\def\PY@tc##1{\textcolor[rgb]{0.25,0.50,0.50}{##1}}}
\expandafter\def\csname PY@tok@cpf\endcsname{\let\PY@it=\textit\def\PY@tc##1{\textcolor[rgb]{0.25,0.50,0.50}{##1}}}
\expandafter\def\csname PY@tok@c1\endcsname{\let\PY@it=\textit\def\PY@tc##1{\textcolor[rgb]{0.25,0.50,0.50}{##1}}}
\expandafter\def\csname PY@tok@cs\endcsname{\let\PY@it=\textit\def\PY@tc##1{\textcolor[rgb]{0.25,0.50,0.50}{##1}}}

\def\PYZbs{\char`\\}
\def\PYZus{\char`\_}
\def\PYZob{\char`\{}
\def\PYZcb{\char`\}}
\def\PYZca{\char`\^}
\def\PYZam{\char`\&}
\def\PYZlt{\char`\<}
\def\PYZgt{\char`\>}
\def\PYZsh{\char`\#}
\def\PYZpc{\char`\%}
\def\PYZdl{\char`\$}
\def\PYZhy{\char`\-}
\def\PYZsq{\char`\'}
\def\PYZdq{\char`\"}
\def\PYZti{\char`\~}
% for compatibility with earlier versions
\def\PYZat{@}
\def\PYZlb{[}
\def\PYZrb{]}
\makeatother


    % Exact colors from NB
    \definecolor{incolor}{rgb}{0.0, 0.0, 0.5}
    \definecolor{outcolor}{rgb}{0.545, 0.0, 0.0}



    
    % Prevent overflowing lines due to hard-to-break entities
    \sloppy 
    % Setup hyperref package
    \hypersetup{
      breaklinks=true,  % so long urls are correctly broken across lines
      colorlinks=true,
      urlcolor=urlcolor,
      linkcolor=linkcolor,
      citecolor=citecolor,
      }
    % Slightly bigger margins than the latex defaults
    
    \geometry{verbose,tmargin=1in,bmargin=1in,lmargin=1in,rmargin=1in}
    
    

    \begin{document}
    
    
    \maketitle
    
    

    
    \subsection{Python primitive data
types}\label{python-primitive-data-types}

    Python provides multiple variables types to hold data within. The type
of the variable is determined by the assignment operation. The
programmer needs not to explicitly declare the variable type. The
following is a list of data types.

    \subsubsection{\texorpdfstring{\textbf{Boolean Data
Type}}{Boolean Data Type}}\label{boolean-data-type}

Variables of this types has only two values, either \texttt{True} or
\texttt{False}. This type of variables is very useful when we need to
determine the control flow of the program depending on a truth value of
some thing. The name boolean comes after George Boole, a great English
mathematician who works on converting logic to mathematical operations.
One branches of math is named after him, Boolean Algebra.

    \subsubsection{\texorpdfstring{\textbf{Numeric
Types}}{Numeric Types}}\label{numeric-types}

This category holds five data types, \texttt{int} for integers,
\texttt{float} for decimal value numbers, and, interestingly,
\texttt{complex} for complex numbers.

The following code gives example of variables of each data type:

    \begin{Verbatim}[commandchars=\\\{\}]
{\color{incolor}In [{\color{incolor}13}]:} \PY{n}{int\PYZus{}variable} \PY{o}{=} \PY{n+nb}{int}\PY{p}{(}\PY{l+m+mi}{5}\PY{p}{)} \PY{c+c1}{\PYZsh{} you can write: intVariable = 5}
         \PY{n}{float\PYZus{}variable} \PY{o}{=} \PY{n+nb}{float}\PY{p}{(}\PY{o}{\PYZhy{}}\PY{l+m+mf}{5.5}\PY{p}{)} \PY{c+c1}{\PYZsh{} you can write: floatVariable = 5.5}
         \PY{n}{complex\PYZus{}variable} \PY{o}{=} \PY{n+nb}{complex}\PY{p}{(}\PY{l+m+mi}{5}\PY{p}{,}\PY{l+m+mi}{4}\PY{p}{)}
         
         \PY{n+nb}{print}\PY{p}{(}\PY{n}{int\PYZus{}variable}\PY{p}{)}
         \PY{n+nb}{print}\PY{p}{(}\PY{n}{float\PYZus{}variable}\PY{p}{)}
         \PY{n+nb}{print}\PY{p}{(}\PY{n}{complex\PYZus{}variable}\PY{p}{)}
\end{Verbatim}


    \begin{Verbatim}[commandchars=\\\{\}]
5
-5.5
(5+4j)

    \end{Verbatim}

    You can convert between these data types as well. However, keep in mind
that converting from float values to integer values will truncate the
decimal part of the number.

    \begin{Verbatim}[commandchars=\\\{\}]
{\color{incolor}In [{\color{incolor}14}]:} \PY{n}{to\PYZus{}float} \PY{o}{=} \PY{n+nb}{float}\PY{p}{(}\PY{n}{int\PYZus{}variable}\PY{p}{)}
         \PY{n+nb}{print}\PY{p}{(}\PY{n}{to\PYZus{}float}\PY{p}{)}
         \PY{n}{to\PYZus{}int} \PY{o}{=} \PY{n+nb}{int}\PY{p}{(}\PY{n}{float\PYZus{}variable}\PY{p}{)}
         \PY{n+nb}{print}\PY{p}{(}\PY{n}{to\PYZus{}int}\PY{p}{)}
         \PY{n}{to\PYZus{}complex} \PY{o}{=} \PY{n+nb}{complex}\PY{p}{(}\PY{n}{int\PYZus{}variable}\PY{p}{,} \PY{n}{float\PYZus{}variable}\PY{p}{)}
         \PY{n+nb}{print}\PY{p}{(}\PY{n}{to\PYZus{}complex}\PY{p}{)}
\end{Verbatim}


    \begin{Verbatim}[commandchars=\\\{\}]
5.0
-5
(5-5.5j)

    \end{Verbatim}

    There are plenty of mathematical operations you can do on numbers. The
following are just examples. \textbf{The table is taken from:
https://www.programiz.com/python-programming/modules/math}. \emph{Some
functions in the original table are not in this table. Go there for more
functions}

    \begin{verbatim}
<caption>List of Functions in Python Math Module</caption>
<tbody>
    <tr>
        <th style="text-align:left">Function</th>
        <th style="text-align:left">Description</th>
    </tr>
    <tr>
        <td style="text-align:left">ceil(x)</td>
        <td style="text-align:left">Returns the smallest integer greater than or equal to x.</td>
    </tr>
    <tr>
        <td style="text-align:left">copysign(x, y)</td>
        <td style="text-align:left">Returns x with the sign of y</td>
    </tr>
    <tr>
        <td style="text-align:left">fabs(x)</td>
        <td style="text-align:left">Returns the absolute value of x</td>
    </tr>
    <tr>
        <td style="text-align:left">factorial(x)</td>
        <td style="text-align:left">Returns the factorial of x</td>
    </tr>
    <tr>
        <td style="text-align:left">floor(x)</td>
        <td style="text-align:left">Returns the largest integer less than or equal to x</td>
    </tr>
    <tr>
        <td style="text-align:left">fmod(x, y)</td>
        <td style="text-align:left">Returns the remainder when x is divided by y</td>
    </tr>
    <tr>
        <td style="text-align:left">isinf(x)</td>
        <td style="text-align:left">Returns True if x is a positive or negative infinity</td>
    </tr>
    <tr>
        <td style="text-align:left">modf(x)</td>
        <td style="text-align:left">Returns the fractional and integer parts of x</td>
    </tr>
    <tr>
        <td style="text-align:left">trunc(x)</td>
        <td style="text-align:left">Returns the truncated integer value of x</td>
    </tr>
    <tr>
        <td style="text-align:left">exp(x)</td>
        <td style="text-align:left">Returns e**x</td>
    </tr>
    <tr>
        <td style="text-align:left">log(x[, base])</td>
        <td style="text-align:left">Returns the logarithm of x to the base (defaults to e)</td>
    </tr>
    <tr>
        <td style="text-align:left">log2(x)</td>
        <td style="text-align:left">Returns the base-2 logarithm of x</td>
    </tr>
    <tr>
        <td style="text-align:left">log10(x)</td>
        <td style="text-align:left">Returns the base-10 logarithm of x</td>
    </tr>
    <tr>
        <td style="text-align:left">pow(x, y)</td>
        <td style="text-align:left">Returns x raised to the power y</td>
    </tr>
    <tr>
        <td style="text-align:left">sqrt(x)</td>
        <td style="text-align:left">Returns the square root of x</td>
    </tr>
    <tr>
        <td style="text-align:left">acos(x)</td>
        <td style="text-align:left">Returns the arc cosine of x</td>
    </tr>
    <tr>
        <td style="text-align:left">asin(x)</td>
        <td style="text-align:left">Returns the arc sine of x</td>
    </tr>
    <tr>
        <td style="text-align:left">atan(x)</td>
        <td style="text-align:left">Returns the arc tangent of x</td>
    </tr>
    <tr>
        <td style="text-align:left">atan2(y, x)</td>
        <td style="text-align:left">Returns atan(y / x)</td>
    </tr>
    <tr>
        <td style="text-align:left">cos(x)</td>
        <td style="text-align:left">Returns the cosine of x</td>
    </tr>
    <tr>
        <td style="text-align:left">sin(x)</td>
        <td style="text-align:left">Returns the sine of x</td>
    </tr>
    <tr>
        <td style="text-align:left">tan(x)</td>
        <td style="text-align:left">Returns the tangent of x</td>
    </tr>
    <tr>
        <td style="text-align:left">degrees(x)</td>
        <td style="text-align:left">Converts angle x from radians to degrees</td>
    </tr>
    <tr>
        <td style="text-align:left">radians(x)</td>
        <td style="text-align:left">Converts angle x from degrees to radians</td>
    </tr>
    <tr>
        <td style="text-align:left">acosh(x)</td>
        <td style="text-align:left">Returns the inverse hyperbolic cosine of x</td>
    </tr>
    <tr>
        <td style="text-align:left">asinh(x)</td>
        <td style="text-align:left">Returns the inverse hyperbolic sine of x</td>
    </tr>
    <tr>
        <td style="text-align:left">atanh(x)</td>
        <td style="text-align:left">Returns the inverse hyperbolic tangent of x</td>
    </tr>
    <tr>
        <td style="text-align:left">cosh(x)</td>
        <td style="text-align:left">Returns the hyperbolic cosine of x</td>
    </tr>
    <tr>
        <td style="text-align:left">sinh(x)</td>
        <td style="text-align:left">Returns the hyperbolic cosine of x</td>
    </tr>
    <tr>
        <td style="text-align:left">tanh(x)</td>
        <td style="text-align:left">Returns the hyperbolic tangent of x</td>
    </tr>
    <tr>
        <td style="text-align:left">erf(x)</td>
        <td style="text-align:left">Returns the error function at x</td>
    </tr>
    <tr>
        <td style="text-align:left">erfc(x)</td>
        <td style="text-align:left">Returns the complementary error function at x</td>
    </tr>
    <tr>
        <td style="text-align:left">gamma(x)</td>
        <td style="text-align:left">Returns the Gamma function at x</td>
    </tr>
    <tr>
        <td style="text-align:left">lgamma(x)</td>
        <td style="text-align:left">Returns the natural logarithm of the absolute value of the Gamma function at x</td>
    </tr>
    <tr>
        <td style="text-align:left">pi</td>
        <td style="text-align:left">Mathematical constant, the ratio of circumference of a circle to it&#39;s diameter (3.14159...)</td>
    </tr>
    <tr>
        <td style="text-align:left">e</td>
        <td style="text-align:left">mathematical constant e (2.71828...)</td>
    </tr>
</tbody>
\end{verbatim}

    \begin{Verbatim}[commandchars=\\\{\}]
{\color{incolor}In [{\color{incolor}15}]:} \PY{c+c1}{\PYZsh{} Most of these functions are in the math module.}
         \PY{c+c1}{\PYZsh{} we need to import that moduel in order to use them.}
         \PY{k+kn}{import} \PY{n+nn}{math}
         \PY{n}{x} \PY{o}{=} \PY{l+m+mi}{2048}
         \PY{n}{y} \PY{o}{=} \PY{l+m+mi}{7}
         \PY{n}{z} \PY{o}{=} \PY{l+m+mf}{1.25}
         \PY{n}{w} \PY{o}{=} \PY{l+m+mf}{1.75}
         
         \PY{n+nb}{print}\PY{p}{(}\PY{n}{math}\PY{o}{.}\PY{n}{ceil}\PY{p}{(}\PY{n}{z}\PY{p}{)}\PY{p}{)}
         \PY{n+nb}{print}\PY{p}{(}\PY{n}{math}\PY{o}{.}\PY{n}{floor}\PY{p}{(}\PY{n}{z}\PY{p}{)}\PY{p}{)}
         \PY{n+nb}{print}\PY{p}{(}\PY{n}{math}\PY{o}{.}\PY{n}{modf}\PY{p}{(}\PY{n}{z}\PY{p}{)}\PY{p}{)}
         \PY{n+nb}{print}\PY{p}{(}\PY{n}{math}\PY{o}{.}\PY{n}{factorial}\PY{p}{(}\PY{n}{y}\PY{p}{)}\PY{p}{)}
         \PY{n+nb}{print}\PY{p}{(}\PY{n}{math}\PY{o}{.}\PY{n}{log2}\PY{p}{(}\PY{n}{x}\PY{p}{)}\PY{p}{)}
         \PY{n+nb}{print}\PY{p}{(}\PY{n}{math}\PY{o}{.}\PY{n}{exp}\PY{p}{(}\PY{n}{math}\PY{o}{.}\PY{n}{e}\PY{p}{)}\PY{p}{)} \PY{c+c1}{\PYZsh{} e\PYZca{}e}
         \PY{n+nb}{print}\PY{p}{(}\PY{n}{math}\PY{o}{.}\PY{n}{pi}\PY{o}{*}\PY{n}{math}\PY{o}{.}\PY{n}{pow}\PY{p}{(}\PY{l+m+mi}{7}\PY{p}{,}\PY{l+m+mi}{2}\PY{p}{)}\PY{p}{)} \PY{c+c1}{\PYZsh{} pi * 7\PYZca{}2}
         \PY{n+nb}{print}\PY{p}{(}\PY{n+nb}{round}\PY{p}{(}\PY{n}{z}\PY{p}{)}\PY{p}{)}
         \PY{n+nb}{print}\PY{p}{(}\PY{n+nb}{round}\PY{p}{(}\PY{n}{w}\PY{p}{)}\PY{p}{)}
         \PY{n+nb}{print}\PY{p}{(}\PY{n}{math}\PY{o}{.}\PY{n}{sqrt}\PY{p}{(}\PY{n}{w}\PY{p}{)}\PY{p}{)}
\end{Verbatim}


    \begin{Verbatim}[commandchars=\\\{\}]
2
1
(0.25, 1.0)
5040
11.0
15.154262241479262
153.93804002589985
1
2
1.3228756555322954

    \end{Verbatim}

    \subsubsection{\texorpdfstring{\textbf{Strings}}{Strings}}\label{strings}

    Strings are the data type representing a sequence of characters. Almost
every programming language has its way of manipulating strings and
python is not an exception.

As we have with numeric data types, strings have their own functions and
methods. In this section, we will try to delve a little bit on string
functions and methods.

    \begin{Verbatim}[commandchars=\\\{\}]
{\color{incolor}In [{\color{incolor}20}]:} \PY{c+c1}{\PYZsh{} Strings can be defined double quoted as follows:}
         \PY{n}{a} \PY{o}{=} \PY{l+s+s2}{\PYZdq{}}\PY{l+s+s2}{I am a string}\PY{l+s+s2}{\PYZdq{}}
         \PY{c+c1}{\PYZsh{} or single quoted as follows:}
         \PY{n}{b} \PY{o}{=} \PY{l+s+s1}{\PYZsq{}}\PY{l+s+s1}{I am another string!}\PY{l+s+s1}{\PYZsq{}}
         \PY{n+nb}{print}\PY{p}{(}\PY{n}{a}\PY{p}{)}
         \PY{n+nb}{print}\PY{p}{(}\PY{n}{b}\PY{p}{)}
\end{Verbatim}


    \begin{Verbatim}[commandchars=\\\{\}]
I am a string
I am another string!

    \end{Verbatim}

    \paragraph{String functions and
properties}\label{string-functions-and-properties}

    \begin{itemize}
\tightlist
\item
  \textbf{Strings can be compared}
\end{itemize}

    By comparing tow strings, the alphabetically greater string will be
returned. Look at the following:

    \begin{Verbatim}[commandchars=\\\{\}]
{\color{incolor}In [{\color{incolor}16}]:} \PY{n}{a} \PY{o}{=} \PY{l+s+s2}{\PYZdq{}}\PY{l+s+s2}{a}\PY{l+s+s2}{\PYZdq{}}
         \PY{n}{b} \PY{o}{=} \PY{l+s+s1}{\PYZsq{}}\PY{l+s+s1}{b}\PY{l+s+s1}{\PYZsq{}}
         \PY{n+nb}{print} \PY{p}{(}\PY{n}{a}\PY{o}{\PYZgt{}}\PY{n}{b}\PY{p}{)}
         
         \PY{n}{a} \PY{o}{=} \PY{l+s+s2}{\PYZdq{}}\PY{l+s+s2}{ai}\PY{l+s+s2}{\PYZdq{}}
         \PY{n}{b} \PY{o}{=} \PY{l+s+s2}{\PYZdq{}}\PY{l+s+s2}{ak}\PY{l+s+s2}{\PYZdq{}}
         \PY{n+nb}{print}\PY{p}{(}\PY{n}{a}\PY{o}{\PYZlt{}}\PY{n}{b}\PY{p}{)}
\end{Verbatim}


    \begin{Verbatim}[commandchars=\\\{\}]
False
True

    \end{Verbatim}

    \begin{itemize}
\tightlist
\item
  \textbf{Length of the string} \texttt{len()} function will return the
  number of characters in a given string. This function is very useful!
\end{itemize}

    \begin{Verbatim}[commandchars=\\\{\}]
{\color{incolor}In [{\color{incolor}17}]:} \PY{n}{a} \PY{o}{=} \PY{l+s+s2}{\PYZdq{}}\PY{l+s+s2}{this is string!}\PY{l+s+s2}{\PYZdq{}} \PY{c+c1}{\PYZsh{} there are 15 characters!! dont forget to count spaces :)}
         \PY{n+nb}{print}\PY{p}{(}\PY{n+nb}{len}\PY{p}{(}\PY{n}{a}\PY{p}{)}\PY{p}{)}
\end{Verbatim}


    \begin{Verbatim}[commandchars=\\\{\}]
15

    \end{Verbatim}

    \begin{itemize}
\tightlist
\item
  \textbf{String concatenation}
\end{itemize}

    We can put two strings together in one string by just using the '+'
operator. as follows:

    \begin{Verbatim}[commandchars=\\\{\}]
{\color{incolor}In [{\color{incolor}22}]:} \PY{n}{a} \PY{o}{=} \PY{l+s+s2}{\PYZdq{}}\PY{l+s+s2}{This is a string!}\PY{l+s+s2}{\PYZdq{}}
         \PY{c+c1}{\PYZsh{} do not forget to add space in the last of the first concatenated string}
         \PY{c+c1}{\PYZsh{} or in the first of the second string to so that the resulted string looks nice!}
         \PY{n}{b} \PY{o}{=} \PY{l+s+s2}{\PYZdq{}}\PY{l+s+s2}{ and this is a second string!}\PY{l+s+s2}{\PYZdq{}}
         \PY{c+c1}{\PYZsh{} you can have multiple strings in the print statement!!}
         \PY{n+nb}{print}\PY{p}{(}\PY{l+s+s2}{\PYZdq{}}\PY{l+s+s2}{the resulted string is:}\PY{l+s+s2}{\PYZdq{}}\PY{p}{,} \PY{n}{a}\PY{o}{+}\PY{n}{b}\PY{p}{)}
\end{Verbatim}


    \begin{Verbatim}[commandchars=\\\{\}]
the resulted string is: This is a string! and this is a second string!

    \end{Verbatim}

    \begin{itemize}
\tightlist
\item
  \textbf{Indexing}
\end{itemize}

    You can retrieve any character from a string by indexing it in a square
brackets {[} {]}! \textbf{Do not forget that the first character in the
string is indexed by 0.} \emph{Remember that} indexing a string with a
greater, or lower, number will fire a python error!

    \begin{Verbatim}[commandchars=\\\{\}]
{\color{incolor}In [{\color{incolor}27}]:} \PY{c+c1}{\PYZsh{} the value of variable a is \PYZdq{}this is string\PYZdq{} from the previous runnable cell.}
         \PY{n+nb}{print}\PY{p}{(}\PY{l+s+s2}{\PYZdq{}}\PY{l+s+s2}{the sixth character of }\PY{l+s+s2}{\PYZsq{}}\PY{l+s+s2}{a}\PY{l+s+s2}{\PYZsq{}}\PY{l+s+s2}{ string is:}\PY{l+s+s2}{\PYZdq{}}\PY{p}{,} \PY{n}{a}\PY{p}{[}\PY{l+m+mi}{5}\PY{p}{]}\PY{p}{)}
         \PY{n+nb}{print}\PY{p}{(}\PY{l+s+s2}{\PYZdq{}}\PY{l+s+s2}{the first character of }\PY{l+s+s2}{\PYZsq{}}\PY{l+s+s2}{a}\PY{l+s+s2}{\PYZsq{}}\PY{l+s+s2}{ string is:}\PY{l+s+s2}{\PYZdq{}}\PY{p}{,} \PY{n}{a}\PY{p}{[}\PY{l+m+mi}{0}\PY{p}{]}\PY{p}{)}
         
         \PY{c+c1}{\PYZsh{} here we will print the last element of the string.}
         \PY{c+c1}{\PYZsh{} by using len() function, it is not necessary to know the length of the string.}
         \PY{c+c1}{\PYZsh{} we need to subtract it by one becuase of the first index which is ZERO.}
         \PY{n+nb}{print}\PY{p}{(}\PY{l+s+s2}{\PYZdq{}}\PY{l+s+s2}{The last character of }\PY{l+s+s2}{\PYZsq{}}\PY{l+s+s2}{a}\PY{l+s+s2}{\PYZsq{}}\PY{l+s+s2}{ is: }\PY{l+s+s2}{\PYZdq{}}\PY{o}{+}\PY{n}{a}\PY{p}{[}\PY{n+nb}{len}\PY{p}{(}\PY{n}{a}\PY{p}{)}\PY{o}{\PYZhy{}}\PY{l+m+mi}{1}\PY{p}{]}\PY{p}{)}
         
         \PY{c+c1}{\PYZsh{} We may use floor() function with len() to get the middle element of the string as well}
         \PY{c+c1}{\PYZsh{} remember that we cannot use float \PYZdq{}real\PYZdq{} numbers to index string. Integers are only allowed.}
         \PY{n+nb}{print}\PY{p}{(}\PY{l+s+s2}{\PYZdq{}}\PY{l+s+s2}{the middle character of }\PY{l+s+s2}{\PYZsq{}}\PY{l+s+s2}{a}\PY{l+s+s2}{\PYZsq{}}\PY{l+s+s2}{ is:}\PY{l+s+s2}{\PYZdq{}}\PY{o}{+}\PY{n}{a}\PY{p}{[}\PY{n}{math}\PY{o}{.}\PY{n}{floor}\PY{p}{(}\PY{n+nb}{len}\PY{p}{(}\PY{n}{a}\PY{p}{)}\PY{o}{/}\PY{l+m+mi}{2}\PY{p}{)}\PY{p}{]}\PY{p}{)}
         
         \PY{c+c1}{\PYZsh{} this code will fire an error:}
         \PY{n+nb}{print}\PY{p}{(}\PY{n}{a}\PY{p}{[}\PY{n+nb}{len}\PY{p}{(}\PY{n}{a}\PY{p}{)}\PY{p}{]}\PY{p}{)}
\end{Verbatim}


    \begin{Verbatim}[commandchars=\\\{\}]
the sixth character of 'a' string is: i
the first character of 'a' string is: t
The last character of 'a' is: g
the middle character of 'a' is:a

    \end{Verbatim}

    \begin{Verbatim}[commandchars=\\\{\}]

        ---------------------------------------------------------------------------

        IndexError                                Traceback (most recent call last)

        <ipython-input-27-6ba032e945d5> in <module>()
         13 
         14 \# this code will fire an error:
    ---> 15 print(a[len(a)])
    

        IndexError: string index out of range

    \end{Verbatim}

    \textbf{Interestingly, } Python allows for negative indexing. What does
this mean? It means that we can use negative numbers to retrieve
characters from strings. Negative indexes will start from the last. Thus
index of -1 means the last character of the string and -2 is the second
to last and so on.

    \begin{Verbatim}[commandchars=\\\{\}]
{\color{incolor}In [{\color{incolor}28}]:} \PY{n}{a} \PY{o}{=} \PY{l+s+s2}{\PYZdq{}}\PY{l+s+s2}{this is a string}\PY{l+s+s2}{\PYZdq{}}
         \PY{n+nb}{print}\PY{p}{(}\PY{l+s+s2}{\PYZdq{}}\PY{l+s+s2}{the last char is:}\PY{l+s+s2}{\PYZdq{}}\PY{p}{,} \PY{n}{a}\PY{p}{[}\PY{o}{\PYZhy{}}\PY{l+m+mi}{1}\PY{p}{]}\PY{p}{)}
         \PY{n+nb}{print}\PY{p}{(}\PY{l+s+s2}{\PYZdq{}}\PY{l+s+s2}{the second to last char is:}\PY{l+s+s2}{\PYZdq{}}\PY{p}{,} \PY{n}{a}\PY{p}{[}\PY{o}{\PYZhy{}}\PY{l+m+mi}{2}\PY{p}{]}\PY{p}{)}
\end{Verbatim}


    \begin{Verbatim}[commandchars=\\\{\}]
the last char is: g
the second to last char is: n

    \end{Verbatim}

    The following picture is taken from Google developers website and gives
a good description about the indexes. I just copy and paste the image
into the following cell.

    \begin{figure}
\centering
\includegraphics{attachment:image.png}
\caption{image.png}
\end{figure}

    \begin{itemize}
\tightlist
\item
  \textbf{String slicing} String slicing is the python method to
  retrieve a substring of a given string. As a casual indexing, we can
  use square brackets to slice a string as follows:
\end{itemize}

\textbf{str.{[}start: stop: step{]}}

\textbf{start}: the position of the first character to retrieve.

\textbf{stop}: the position of the last character to retrieve.

\textbf{step}: how many steps to take.

    \begin{Verbatim}[commandchars=\\\{\}]
{\color{incolor}In [{\color{incolor}40}]:} \PY{n+nb}{print}\PY{p}{(}\PY{l+s+s1}{\PYZsq{}}\PY{l+s+s1}{a is:}\PY{l+s+s1}{\PYZsq{}}\PY{p}{,}\PY{n}{a}\PY{p}{)}
         
         \PY{n+nb}{print}\PY{p}{(}\PY{l+s+s1}{\PYZsq{}}\PY{l+s+s1}{starts from the third char to seventh char.}\PY{l+s+s1}{\PYZsq{}}\PY{p}{)}
         \PY{n+nb}{print}\PY{p}{(}\PY{n}{a}\PY{p}{[}\PY{l+m+mi}{2}\PY{p}{:}\PY{l+m+mi}{6}\PY{p}{]}\PY{p}{)}
         
         \PY{n+nb}{print}\PY{p}{(}\PY{l+s+s1}{\PYZsq{}}\PY{l+s+s1}{starts from the second char to seventh char stepping two chars at a time.}\PY{l+s+s1}{\PYZsq{}}\PY{p}{)}
         \PY{n+nb}{print}\PY{p}{(}\PY{n}{a}\PY{p}{[}\PY{l+m+mi}{2}\PY{p}{:}\PY{l+m+mi}{6}\PY{p}{:}\PY{l+m+mi}{2}\PY{p}{]}\PY{p}{)}
         
         \PY{c+c1}{\PYZsh{}this will return the whole string :)}
         \PY{n+nb}{print}\PY{p}{(}\PY{l+s+s2}{\PYZdq{}}\PY{l+s+s2}{the whole }\PY{l+s+s2}{\PYZsq{}}\PY{l+s+s2}{a}\PY{l+s+s2}{\PYZsq{}}\PY{l+s+s2}{ string is:}\PY{l+s+s2}{\PYZdq{}}\PY{p}{,} \PY{n}{a}\PY{p}{[}\PY{l+m+mi}{0}\PY{p}{:}\PY{n+nb}{len}\PY{p}{(}\PY{n}{a}\PY{p}{)}\PY{p}{]}\PY{p}{)}
\end{Verbatim}


    \begin{Verbatim}[commandchars=\\\{\}]
a is: this is a string
starts from the third char to seventh char.
is i
starts from the second char to seventh char stepping two chars at a time.
i 
the whole 'a' string is: this is a string

    \end{Verbatim}

    \begin{itemize}
\tightlist
\item
  \textbf{More on slicing}
\end{itemize}

    Many things we can do with strings. - we can leave indexes empty so that
python can use the default values. (i.e {[} : : {]}). - defalut value
for \textbf{start} is 0. - default value for \textbf{stop} is
\emph{len()}. - default value for \textbf{step} is 1. - negative indexes
are allowed. - {[} : : -1{]} will flip the defaults to: {[}-1:
-(len()+1):-1{]}. Simply, it will reverse the string. - If negative
numbers are used in the \textbf{step}, then \textbf{start} must be
greater than stop and the other way around for positive numbers.

    \begin{Verbatim}[commandchars=\\\{\}]
{\color{incolor}In [{\color{incolor}43}]:} \PY{n+nb}{print}\PY{p}{(}\PY{l+s+s2}{\PYZdq{}}\PY{l+s+s2}{this is }\PY{l+s+s2}{\PYZsq{}}\PY{l+s+s2}{a}\PY{l+s+s2}{\PYZsq{}}\PY{l+s+s2}{ string:}\PY{l+s+s2}{\PYZdq{}}\PY{p}{,} \PY{n}{a}\PY{p}{)}
         
         \PY{c+c1}{\PYZsh{} using default values will return the whole string:}
         \PY{n+nb}{print}\PY{p}{(}\PY{n}{a}\PY{p}{[}\PY{p}{:}\PY{p}{:}\PY{p}{]}\PY{p}{)}
         
         \PY{c+c1}{\PYZsh{} reverse the string:}
         \PY{n+nb}{print}\PY{p}{(}\PY{n}{a}\PY{p}{[}\PY{p}{:}\PY{p}{:}\PY{o}{\PYZhy{}}\PY{l+m+mi}{1}\PY{p}{]}\PY{p}{)}
\end{Verbatim}


    \begin{Verbatim}[commandchars=\\\{\}]
this is 'a' string: this is a string
this is a string
gnirts a si siht

    \end{Verbatim}

    \paragraph{More python string
functions.}\label{more-python-string-functions.}

    This section, will talk about general strings functions. The description
of each method is given within the code block in the following coding
paragraph.

    \begin{Verbatim}[commandchars=\\\{\}]
{\color{incolor}In [{\color{incolor}89}]:} \PY{n}{a} \PY{o}{=} \PY{l+s+s2}{\PYZdq{}}\PY{l+s+s2}{This is a string}\PY{l+s+s2}{\PYZdq{}}
         
         \PY{c+c1}{\PYZsh{} the following functions are self\PYZhy{}descriptive functions.}
         \PY{n}{a} \PY{o}{=} \PY{n}{a}\PY{o}{.}\PY{n}{upper}\PY{p}{(}\PY{p}{)}
         \PY{n+nb}{print}\PY{p}{(}\PY{l+s+s2}{\PYZdq{}}\PY{l+s+s2}{1:}\PY{l+s+s2}{\PYZdq{}}\PY{p}{,}\PY{n}{a}\PY{p}{)}
         
         \PY{n}{a} \PY{o}{=} \PY{n}{a}\PY{o}{.}\PY{n}{lower}\PY{p}{(}\PY{p}{)}
         \PY{n+nb}{print}\PY{p}{(}\PY{l+s+s2}{\PYZdq{}}\PY{l+s+s2}{2:}\PY{l+s+s2}{\PYZdq{}}\PY{p}{,}\PY{n}{a}\PY{p}{)}
         
         
         \PY{c+c1}{\PYZsh{} the following functions will test the given string according to their namings.}
         
         \PY{c+c1}{\PYZsh{}This function test whether a string is all alphabets.}
         \PY{c+c1}{\PYZsh{}Spaces and !! will cause the function to return False.}
         \PY{n+nb}{print}\PY{p}{(}\PY{l+s+s2}{\PYZdq{}}\PY{l+s+s2}{3:}\PY{l+s+s2}{\PYZdq{}}\PY{p}{,}\PY{n}{a}\PY{o}{.}\PY{n}{isalpha}\PY{p}{(}\PY{p}{)}\PY{p}{)}
         
         \PY{c+c1}{\PYZsh{}By the way, You can define a string on the fly and test it against this function.}
         \PY{c+c1}{\PYZsh{}Look at this code:}
         \PY{n+nb}{print}\PY{p}{(}\PY{l+s+s2}{\PYZdq{}}\PY{l+s+s2}{4:}\PY{l+s+s2}{\PYZdq{}}\PY{p}{,}\PY{l+s+s2}{\PYZdq{}}\PY{l+s+s2}{thisstringisallalphabetics}\PY{l+s+s2}{\PYZdq{}}\PY{o}{.}\PY{n}{isalpha}\PY{p}{(}\PY{p}{)}\PY{p}{)}
         
         \PY{c+c1}{\PYZsh{}tests for digits:}
         \PY{n+nb}{print}\PY{p}{(}\PY{l+s+s2}{\PYZdq{}}\PY{l+s+s2}{5:}\PY{l+s+s2}{\PYZdq{}}\PY{p}{,}\PY{n}{a}\PY{o}{.}\PY{n}{isdigit}\PY{p}{(}\PY{p}{)}\PY{p}{)}
         \PY{n+nb}{print}\PY{p}{(}\PY{l+s+s2}{\PYZdq{}}\PY{l+s+s2}{6:}\PY{l+s+s2}{\PYZdq{}}\PY{p}{,}\PY{l+s+s2}{\PYZdq{}}\PY{l+s+s2}{123456789}\PY{l+s+s2}{\PYZdq{}}\PY{o}{.}\PY{n}{isdigit}\PY{p}{(}\PY{p}{)}\PY{p}{)}
         
         \PY{c+c1}{\PYZsh{}tests for space:}
         \PY{n+nb}{print}\PY{p}{(}\PY{l+s+s2}{\PYZdq{}}\PY{l+s+s2}{7:}\PY{l+s+s2}{\PYZdq{}}\PY{p}{,}\PY{n}{a}\PY{o}{.}\PY{n}{isspace}\PY{p}{(}\PY{p}{)}\PY{p}{)}
         \PY{n+nb}{print}\PY{p}{(}\PY{l+s+s2}{\PYZdq{}}\PY{l+s+s2}{8:}\PY{l+s+s2}{\PYZdq{}}\PY{p}{,}\PY{l+s+s2}{\PYZdq{}}\PY{l+s+s2}{  }\PY{l+s+s2}{\PYZdq{}}\PY{o}{.}\PY{n}{isspace}\PY{p}{(}\PY{p}{)}\PY{p}{)}
         
         
         \PY{c+c1}{\PYZsh{} this function will search a string for a given substring.}
         \PY{c+c1}{\PYZsh{} it returns the index of teh first occurance.}
         \PY{n+nb}{print}\PY{p}{(}\PY{l+s+s2}{\PYZdq{}}\PY{l+s+s2}{9:}\PY{l+s+s2}{\PYZdq{}}\PY{p}{,}\PY{n}{a}\PY{o}{.}\PY{n}{find}\PY{p}{(}\PY{l+s+s2}{\PYZdq{}}\PY{l+s+s2}{is}\PY{l+s+s2}{\PYZdq{}}\PY{p}{)}\PY{p}{)}
         
         \PY{c+c1}{\PYZsh{} this function will replace a given substring with another substring.}
         
         \PY{n}{a} \PY{o}{=} \PY{n}{a}\PY{o}{.}\PY{n}{replace}\PY{p}{(}\PY{l+s+s2}{\PYZdq{}}\PY{l+s+s2}{string}\PY{l+s+s2}{\PYZdq{}}\PY{p}{,}\PY{l+s+s2}{\PYZdq{}}\PY{l+s+s2}{nicer string!!}\PY{l+s+s2}{\PYZdq{}}\PY{p}{)}
         \PY{n+nb}{print}\PY{p}{(}\PY{l+s+s2}{\PYZdq{}}\PY{l+s+s2}{10:}\PY{l+s+s2}{\PYZdq{}}\PY{p}{,}\PY{n}{a}\PY{p}{)}
         
         \PY{c+c1}{\PYZsh{}}
\end{Verbatim}


    \begin{Verbatim}[commandchars=\\\{\}]
1: THIS IS A STRING
2: this is a string
3: False
4: True
5: False
6: True
7: False
8: True
9: 2
10: this is a nicer string!!

    \end{Verbatim}

    And by this ends the String section.

    \subsection{Python Operators}\label{python-operators}

    Operators are special characters with specific functionalities. There
are different types of operands. The following lists will show many
operands of different kinds.

    \begin{verbatim}
<caption>Arithmetic operators in Python</caption>
<tbody>
    <tr>
        <th>Operator</th>
        <th>Meaning</th>
        <th>Example</th>
    </tr>
    <tr>
        <td>+</td>
        <td>Add two operands or unary plus</td>
        <td>x + y<br />
            +2</td>
    </tr>
    <tr>
        <td>-</td>
        <td>Subtract right operand from the left or unary minus</td>
        <td>x - y<br />
            -2</td>
    </tr>
    <tr>
        <td>\*</td>
        <td>Multiply two operands</td>
        <td>x \* y</td>
    </tr>
    <tr>
        <td>/</td>
        <td>Divide left operand by the right one (always results into float)</td>
        <td>x / y</td>
    </tr>
    <tr>
        <td>%</td>
        <td>Modulus - remainder of the division of left operand by the right</td>
        <td>x % y (remainder of x/y)</td>
    </tr>
    <tr>
        <td>//</td>
        <td>Floor division - division that results into whole number adjusted to the left in the number line</td>
        <td>x // y</td>
    </tr>
    <tr>
        <td>**</td>
        <td>Exponent - left operand raised to the power of right</td>
        <td>x**y (x to the power y)</td>
    </tr>
</tbody>
\end{verbatim}

    https://www.tutorialspoint.com/python/python\_basic\_operators.htm

    \subsection{Python Lists and Dicts}\label{python-lists-and-dicts}

    https://developers.google.com/edu/python/lists

    https://developers.google.com/edu/python/dict-files

    \subsection{Python Control flow}\label{python-control-flow}

    If statement.

    \subsection{Python Branching}\label{python-branching}

    http://www.pythonforbeginners.com/control-flow-2/python-for-and-while-loops

    \subsection{Resources}\label{resources}

    \begin{itemize}
\item
  \href{https://en.wikipedia.org/wiki/George_Boole}{George Boole}, Link:
  https://en.wikipedia.org/wiki/George\_Boole
\item
  \href{https://docs.python.org/2/library/stdtypes.html}{Python official
  documentation}, Link: https://docs.python.org/2/library/stdtypes.html
\item
  \href{https://www.tutorialspoint.com/python3/python_numbers.htm}{Mathematical
  operations on Numeric Types}, Link:
  https://www.tutorialspoint.com/python3/python\_numbers.htm
\item
  \href{https://www.programiz.com/python-programming/modules/math}{Python
  Mathematical functions},
  Link:https://www.programiz.com/python-programming/modules/math
\end{itemize}


    % Add a bibliography block to the postdoc
    
    
    
    \end{document}
