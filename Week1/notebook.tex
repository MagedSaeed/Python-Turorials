
% Default to the notebook output style

    


% Inherit from the specified cell style.




    
\documentclass[11pt]{article}

    
    
    \usepackage[T1]{fontenc}
    % Nicer default font (+ math font) than Computer Modern for most use cases
    \usepackage{mathpazo}

    % Basic figure setup, for now with no caption control since it's done
    % automatically by Pandoc (which extracts ![](path) syntax from Markdown).
    \usepackage{graphicx}
    % We will generate all images so they have a width \maxwidth. This means
    % that they will get their normal width if they fit onto the page, but
    % are scaled down if they would overflow the margins.
    \makeatletter
    \def\maxwidth{\ifdim\Gin@nat@width>\linewidth\linewidth
    \else\Gin@nat@width\fi}
    \makeatother
    \let\Oldincludegraphics\includegraphics
    % Set max figure width to be 80% of text width, for now hardcoded.
    \renewcommand{\includegraphics}[1]{\Oldincludegraphics[width=.8\maxwidth]{#1}}
    % Ensure that by default, figures have no caption (until we provide a
    % proper Figure object with a Caption API and a way to capture that
    % in the conversion process - todo).
    \usepackage{caption}
    \DeclareCaptionLabelFormat{nolabel}{}
    \captionsetup{labelformat=nolabel}

    \usepackage{adjustbox} % Used to constrain images to a maximum size 
    \usepackage{xcolor} % Allow colors to be defined
    \usepackage{enumerate} % Needed for markdown enumerations to work
    \usepackage{geometry} % Used to adjust the document margins
    \usepackage{amsmath} % Equations
    \usepackage{amssymb} % Equations
    \usepackage{textcomp} % defines textquotesingle
    % Hack from http://tex.stackexchange.com/a/47451/13684:
    \AtBeginDocument{%
        \def\PYZsq{\textquotesingle}% Upright quotes in Pygmentized code
    }
    \usepackage{upquote} % Upright quotes for verbatim code
    \usepackage{eurosym} % defines \euro
    \usepackage[mathletters]{ucs} % Extended unicode (utf-8) support
    \usepackage[utf8x]{inputenc} % Allow utf-8 characters in the tex document
    \usepackage{fancyvrb} % verbatim replacement that allows latex
    \usepackage{grffile} % extends the file name processing of package graphics 
                         % to support a larger range 
    % The hyperref package gives us a pdf with properly built
    % internal navigation ('pdf bookmarks' for the table of contents,
    % internal cross-reference links, web links for URLs, etc.)
    \usepackage{hyperref}
    \usepackage{longtable} % longtable support required by pandoc >1.10
    \usepackage{booktabs}  % table support for pandoc > 1.12.2
    \usepackage[inline]{enumitem} % IRkernel/repr support (it uses the enumerate* environment)
    \usepackage[normalem]{ulem} % ulem is needed to support strikethroughs (\sout)
                                % normalem makes italics be italics, not underlines
    

    
    
    % Colors for the hyperref package
    \definecolor{urlcolor}{rgb}{0,.145,.698}
    \definecolor{linkcolor}{rgb}{.71,0.21,0.01}
    \definecolor{citecolor}{rgb}{.12,.54,.11}

    % ANSI colors
    \definecolor{ansi-black}{HTML}{3E424D}
    \definecolor{ansi-black-intense}{HTML}{282C36}
    \definecolor{ansi-red}{HTML}{E75C58}
    \definecolor{ansi-red-intense}{HTML}{B22B31}
    \definecolor{ansi-green}{HTML}{00A250}
    \definecolor{ansi-green-intense}{HTML}{007427}
    \definecolor{ansi-yellow}{HTML}{DDB62B}
    \definecolor{ansi-yellow-intense}{HTML}{B27D12}
    \definecolor{ansi-blue}{HTML}{208FFB}
    \definecolor{ansi-blue-intense}{HTML}{0065CA}
    \definecolor{ansi-magenta}{HTML}{D160C4}
    \definecolor{ansi-magenta-intense}{HTML}{A03196}
    \definecolor{ansi-cyan}{HTML}{60C6C8}
    \definecolor{ansi-cyan-intense}{HTML}{258F8F}
    \definecolor{ansi-white}{HTML}{C5C1B4}
    \definecolor{ansi-white-intense}{HTML}{A1A6B2}

    % commands and environments needed by pandoc snippets
    % extracted from the output of `pandoc -s`
    \providecommand{\tightlist}{%
      \setlength{\itemsep}{0pt}\setlength{\parskip}{0pt}}
    \DefineVerbatimEnvironment{Highlighting}{Verbatim}{commandchars=\\\{\}}
    % Add ',fontsize=\small' for more characters per line
    \newenvironment{Shaded}{}{}
    \newcommand{\KeywordTok}[1]{\textcolor[rgb]{0.00,0.44,0.13}{\textbf{{#1}}}}
    \newcommand{\DataTypeTok}[1]{\textcolor[rgb]{0.56,0.13,0.00}{{#1}}}
    \newcommand{\DecValTok}[1]{\textcolor[rgb]{0.25,0.63,0.44}{{#1}}}
    \newcommand{\BaseNTok}[1]{\textcolor[rgb]{0.25,0.63,0.44}{{#1}}}
    \newcommand{\FloatTok}[1]{\textcolor[rgb]{0.25,0.63,0.44}{{#1}}}
    \newcommand{\CharTok}[1]{\textcolor[rgb]{0.25,0.44,0.63}{{#1}}}
    \newcommand{\StringTok}[1]{\textcolor[rgb]{0.25,0.44,0.63}{{#1}}}
    \newcommand{\CommentTok}[1]{\textcolor[rgb]{0.38,0.63,0.69}{\textit{{#1}}}}
    \newcommand{\OtherTok}[1]{\textcolor[rgb]{0.00,0.44,0.13}{{#1}}}
    \newcommand{\AlertTok}[1]{\textcolor[rgb]{1.00,0.00,0.00}{\textbf{{#1}}}}
    \newcommand{\FunctionTok}[1]{\textcolor[rgb]{0.02,0.16,0.49}{{#1}}}
    \newcommand{\RegionMarkerTok}[1]{{#1}}
    \newcommand{\ErrorTok}[1]{\textcolor[rgb]{1.00,0.00,0.00}{\textbf{{#1}}}}
    \newcommand{\NormalTok}[1]{{#1}}
    
    % Additional commands for more recent versions of Pandoc
    \newcommand{\ConstantTok}[1]{\textcolor[rgb]{0.53,0.00,0.00}{{#1}}}
    \newcommand{\SpecialCharTok}[1]{\textcolor[rgb]{0.25,0.44,0.63}{{#1}}}
    \newcommand{\VerbatimStringTok}[1]{\textcolor[rgb]{0.25,0.44,0.63}{{#1}}}
    \newcommand{\SpecialStringTok}[1]{\textcolor[rgb]{0.73,0.40,0.53}{{#1}}}
    \newcommand{\ImportTok}[1]{{#1}}
    \newcommand{\DocumentationTok}[1]{\textcolor[rgb]{0.73,0.13,0.13}{\textit{{#1}}}}
    \newcommand{\AnnotationTok}[1]{\textcolor[rgb]{0.38,0.63,0.69}{\textbf{\textit{{#1}}}}}
    \newcommand{\CommentVarTok}[1]{\textcolor[rgb]{0.38,0.63,0.69}{\textbf{\textit{{#1}}}}}
    \newcommand{\VariableTok}[1]{\textcolor[rgb]{0.10,0.09,0.49}{{#1}}}
    \newcommand{\ControlFlowTok}[1]{\textcolor[rgb]{0.00,0.44,0.13}{\textbf{{#1}}}}
    \newcommand{\OperatorTok}[1]{\textcolor[rgb]{0.40,0.40,0.40}{{#1}}}
    \newcommand{\BuiltInTok}[1]{{#1}}
    \newcommand{\ExtensionTok}[1]{{#1}}
    \newcommand{\PreprocessorTok}[1]{\textcolor[rgb]{0.74,0.48,0.00}{{#1}}}
    \newcommand{\AttributeTok}[1]{\textcolor[rgb]{0.49,0.56,0.16}{{#1}}}
    \newcommand{\InformationTok}[1]{\textcolor[rgb]{0.38,0.63,0.69}{\textbf{\textit{{#1}}}}}
    \newcommand{\WarningTok}[1]{\textcolor[rgb]{0.38,0.63,0.69}{\textbf{\textit{{#1}}}}}
    
    
    % Define a nice break command that doesn't care if a line doesn't already
    % exist.
    \def\br{\hspace*{\fill} \\* }
    % Math Jax compatability definitions
    \def\gt{>}
    \def\lt{<}
    % Document parameters
    \title{Introduction to python}
    
    
    

    % Pygments definitions
    
\makeatletter
\def\PY@reset{\let\PY@it=\relax \let\PY@bf=\relax%
    \let\PY@ul=\relax \let\PY@tc=\relax%
    \let\PY@bc=\relax \let\PY@ff=\relax}
\def\PY@tok#1{\csname PY@tok@#1\endcsname}
\def\PY@toks#1+{\ifx\relax#1\empty\else%
    \PY@tok{#1}\expandafter\PY@toks\fi}
\def\PY@do#1{\PY@bc{\PY@tc{\PY@ul{%
    \PY@it{\PY@bf{\PY@ff{#1}}}}}}}
\def\PY#1#2{\PY@reset\PY@toks#1+\relax+\PY@do{#2}}

\expandafter\def\csname PY@tok@w\endcsname{\def\PY@tc##1{\textcolor[rgb]{0.73,0.73,0.73}{##1}}}
\expandafter\def\csname PY@tok@c\endcsname{\let\PY@it=\textit\def\PY@tc##1{\textcolor[rgb]{0.25,0.50,0.50}{##1}}}
\expandafter\def\csname PY@tok@cp\endcsname{\def\PY@tc##1{\textcolor[rgb]{0.74,0.48,0.00}{##1}}}
\expandafter\def\csname PY@tok@k\endcsname{\let\PY@bf=\textbf\def\PY@tc##1{\textcolor[rgb]{0.00,0.50,0.00}{##1}}}
\expandafter\def\csname PY@tok@kp\endcsname{\def\PY@tc##1{\textcolor[rgb]{0.00,0.50,0.00}{##1}}}
\expandafter\def\csname PY@tok@kt\endcsname{\def\PY@tc##1{\textcolor[rgb]{0.69,0.00,0.25}{##1}}}
\expandafter\def\csname PY@tok@o\endcsname{\def\PY@tc##1{\textcolor[rgb]{0.40,0.40,0.40}{##1}}}
\expandafter\def\csname PY@tok@ow\endcsname{\let\PY@bf=\textbf\def\PY@tc##1{\textcolor[rgb]{0.67,0.13,1.00}{##1}}}
\expandafter\def\csname PY@tok@nb\endcsname{\def\PY@tc##1{\textcolor[rgb]{0.00,0.50,0.00}{##1}}}
\expandafter\def\csname PY@tok@nf\endcsname{\def\PY@tc##1{\textcolor[rgb]{0.00,0.00,1.00}{##1}}}
\expandafter\def\csname PY@tok@nc\endcsname{\let\PY@bf=\textbf\def\PY@tc##1{\textcolor[rgb]{0.00,0.00,1.00}{##1}}}
\expandafter\def\csname PY@tok@nn\endcsname{\let\PY@bf=\textbf\def\PY@tc##1{\textcolor[rgb]{0.00,0.00,1.00}{##1}}}
\expandafter\def\csname PY@tok@ne\endcsname{\let\PY@bf=\textbf\def\PY@tc##1{\textcolor[rgb]{0.82,0.25,0.23}{##1}}}
\expandafter\def\csname PY@tok@nv\endcsname{\def\PY@tc##1{\textcolor[rgb]{0.10,0.09,0.49}{##1}}}
\expandafter\def\csname PY@tok@no\endcsname{\def\PY@tc##1{\textcolor[rgb]{0.53,0.00,0.00}{##1}}}
\expandafter\def\csname PY@tok@nl\endcsname{\def\PY@tc##1{\textcolor[rgb]{0.63,0.63,0.00}{##1}}}
\expandafter\def\csname PY@tok@ni\endcsname{\let\PY@bf=\textbf\def\PY@tc##1{\textcolor[rgb]{0.60,0.60,0.60}{##1}}}
\expandafter\def\csname PY@tok@na\endcsname{\def\PY@tc##1{\textcolor[rgb]{0.49,0.56,0.16}{##1}}}
\expandafter\def\csname PY@tok@nt\endcsname{\let\PY@bf=\textbf\def\PY@tc##1{\textcolor[rgb]{0.00,0.50,0.00}{##1}}}
\expandafter\def\csname PY@tok@nd\endcsname{\def\PY@tc##1{\textcolor[rgb]{0.67,0.13,1.00}{##1}}}
\expandafter\def\csname PY@tok@s\endcsname{\def\PY@tc##1{\textcolor[rgb]{0.73,0.13,0.13}{##1}}}
\expandafter\def\csname PY@tok@sd\endcsname{\let\PY@it=\textit\def\PY@tc##1{\textcolor[rgb]{0.73,0.13,0.13}{##1}}}
\expandafter\def\csname PY@tok@si\endcsname{\let\PY@bf=\textbf\def\PY@tc##1{\textcolor[rgb]{0.73,0.40,0.53}{##1}}}
\expandafter\def\csname PY@tok@se\endcsname{\let\PY@bf=\textbf\def\PY@tc##1{\textcolor[rgb]{0.73,0.40,0.13}{##1}}}
\expandafter\def\csname PY@tok@sr\endcsname{\def\PY@tc##1{\textcolor[rgb]{0.73,0.40,0.53}{##1}}}
\expandafter\def\csname PY@tok@ss\endcsname{\def\PY@tc##1{\textcolor[rgb]{0.10,0.09,0.49}{##1}}}
\expandafter\def\csname PY@tok@sx\endcsname{\def\PY@tc##1{\textcolor[rgb]{0.00,0.50,0.00}{##1}}}
\expandafter\def\csname PY@tok@m\endcsname{\def\PY@tc##1{\textcolor[rgb]{0.40,0.40,0.40}{##1}}}
\expandafter\def\csname PY@tok@gh\endcsname{\let\PY@bf=\textbf\def\PY@tc##1{\textcolor[rgb]{0.00,0.00,0.50}{##1}}}
\expandafter\def\csname PY@tok@gu\endcsname{\let\PY@bf=\textbf\def\PY@tc##1{\textcolor[rgb]{0.50,0.00,0.50}{##1}}}
\expandafter\def\csname PY@tok@gd\endcsname{\def\PY@tc##1{\textcolor[rgb]{0.63,0.00,0.00}{##1}}}
\expandafter\def\csname PY@tok@gi\endcsname{\def\PY@tc##1{\textcolor[rgb]{0.00,0.63,0.00}{##1}}}
\expandafter\def\csname PY@tok@gr\endcsname{\def\PY@tc##1{\textcolor[rgb]{1.00,0.00,0.00}{##1}}}
\expandafter\def\csname PY@tok@ge\endcsname{\let\PY@it=\textit}
\expandafter\def\csname PY@tok@gs\endcsname{\let\PY@bf=\textbf}
\expandafter\def\csname PY@tok@gp\endcsname{\let\PY@bf=\textbf\def\PY@tc##1{\textcolor[rgb]{0.00,0.00,0.50}{##1}}}
\expandafter\def\csname PY@tok@go\endcsname{\def\PY@tc##1{\textcolor[rgb]{0.53,0.53,0.53}{##1}}}
\expandafter\def\csname PY@tok@gt\endcsname{\def\PY@tc##1{\textcolor[rgb]{0.00,0.27,0.87}{##1}}}
\expandafter\def\csname PY@tok@err\endcsname{\def\PY@bc##1{\setlength{\fboxsep}{0pt}\fcolorbox[rgb]{1.00,0.00,0.00}{1,1,1}{\strut ##1}}}
\expandafter\def\csname PY@tok@kc\endcsname{\let\PY@bf=\textbf\def\PY@tc##1{\textcolor[rgb]{0.00,0.50,0.00}{##1}}}
\expandafter\def\csname PY@tok@kd\endcsname{\let\PY@bf=\textbf\def\PY@tc##1{\textcolor[rgb]{0.00,0.50,0.00}{##1}}}
\expandafter\def\csname PY@tok@kn\endcsname{\let\PY@bf=\textbf\def\PY@tc##1{\textcolor[rgb]{0.00,0.50,0.00}{##1}}}
\expandafter\def\csname PY@tok@kr\endcsname{\let\PY@bf=\textbf\def\PY@tc##1{\textcolor[rgb]{0.00,0.50,0.00}{##1}}}
\expandafter\def\csname PY@tok@bp\endcsname{\def\PY@tc##1{\textcolor[rgb]{0.00,0.50,0.00}{##1}}}
\expandafter\def\csname PY@tok@fm\endcsname{\def\PY@tc##1{\textcolor[rgb]{0.00,0.00,1.00}{##1}}}
\expandafter\def\csname PY@tok@vc\endcsname{\def\PY@tc##1{\textcolor[rgb]{0.10,0.09,0.49}{##1}}}
\expandafter\def\csname PY@tok@vg\endcsname{\def\PY@tc##1{\textcolor[rgb]{0.10,0.09,0.49}{##1}}}
\expandafter\def\csname PY@tok@vi\endcsname{\def\PY@tc##1{\textcolor[rgb]{0.10,0.09,0.49}{##1}}}
\expandafter\def\csname PY@tok@vm\endcsname{\def\PY@tc##1{\textcolor[rgb]{0.10,0.09,0.49}{##1}}}
\expandafter\def\csname PY@tok@sa\endcsname{\def\PY@tc##1{\textcolor[rgb]{0.73,0.13,0.13}{##1}}}
\expandafter\def\csname PY@tok@sb\endcsname{\def\PY@tc##1{\textcolor[rgb]{0.73,0.13,0.13}{##1}}}
\expandafter\def\csname PY@tok@sc\endcsname{\def\PY@tc##1{\textcolor[rgb]{0.73,0.13,0.13}{##1}}}
\expandafter\def\csname PY@tok@dl\endcsname{\def\PY@tc##1{\textcolor[rgb]{0.73,0.13,0.13}{##1}}}
\expandafter\def\csname PY@tok@s2\endcsname{\def\PY@tc##1{\textcolor[rgb]{0.73,0.13,0.13}{##1}}}
\expandafter\def\csname PY@tok@sh\endcsname{\def\PY@tc##1{\textcolor[rgb]{0.73,0.13,0.13}{##1}}}
\expandafter\def\csname PY@tok@s1\endcsname{\def\PY@tc##1{\textcolor[rgb]{0.73,0.13,0.13}{##1}}}
\expandafter\def\csname PY@tok@mb\endcsname{\def\PY@tc##1{\textcolor[rgb]{0.40,0.40,0.40}{##1}}}
\expandafter\def\csname PY@tok@mf\endcsname{\def\PY@tc##1{\textcolor[rgb]{0.40,0.40,0.40}{##1}}}
\expandafter\def\csname PY@tok@mh\endcsname{\def\PY@tc##1{\textcolor[rgb]{0.40,0.40,0.40}{##1}}}
\expandafter\def\csname PY@tok@mi\endcsname{\def\PY@tc##1{\textcolor[rgb]{0.40,0.40,0.40}{##1}}}
\expandafter\def\csname PY@tok@il\endcsname{\def\PY@tc##1{\textcolor[rgb]{0.40,0.40,0.40}{##1}}}
\expandafter\def\csname PY@tok@mo\endcsname{\def\PY@tc##1{\textcolor[rgb]{0.40,0.40,0.40}{##1}}}
\expandafter\def\csname PY@tok@ch\endcsname{\let\PY@it=\textit\def\PY@tc##1{\textcolor[rgb]{0.25,0.50,0.50}{##1}}}
\expandafter\def\csname PY@tok@cm\endcsname{\let\PY@it=\textit\def\PY@tc##1{\textcolor[rgb]{0.25,0.50,0.50}{##1}}}
\expandafter\def\csname PY@tok@cpf\endcsname{\let\PY@it=\textit\def\PY@tc##1{\textcolor[rgb]{0.25,0.50,0.50}{##1}}}
\expandafter\def\csname PY@tok@c1\endcsname{\let\PY@it=\textit\def\PY@tc##1{\textcolor[rgb]{0.25,0.50,0.50}{##1}}}
\expandafter\def\csname PY@tok@cs\endcsname{\let\PY@it=\textit\def\PY@tc##1{\textcolor[rgb]{0.25,0.50,0.50}{##1}}}

\def\PYZbs{\char`\\}
\def\PYZus{\char`\_}
\def\PYZob{\char`\{}
\def\PYZcb{\char`\}}
\def\PYZca{\char`\^}
\def\PYZam{\char`\&}
\def\PYZlt{\char`\<}
\def\PYZgt{\char`\>}
\def\PYZsh{\char`\#}
\def\PYZpc{\char`\%}
\def\PYZdl{\char`\$}
\def\PYZhy{\char`\-}
\def\PYZsq{\char`\'}
\def\PYZdq{\char`\"}
\def\PYZti{\char`\~}
% for compatibility with earlier versions
\def\PYZat{@}
\def\PYZlb{[}
\def\PYZrb{]}
\makeatother


    % Exact colors from NB
    \definecolor{incolor}{rgb}{0.0, 0.0, 0.5}
    \definecolor{outcolor}{rgb}{0.545, 0.0, 0.0}



    
    % Prevent overflowing lines due to hard-to-break entities
    \sloppy 
    % Setup hyperref package
    \hypersetup{
      breaklinks=true,  % so long urls are correctly broken across lines
      colorlinks=true,
      urlcolor=urlcolor,
      linkcolor=linkcolor,
      citecolor=citecolor,
      }
    % Slightly bigger margins than the latex defaults
    
    \geometry{verbose,tmargin=1in,bmargin=1in,lmargin=1in,rmargin=1in}
    
    

    \begin{document}
    
    
    \maketitle
    
    

    
    Table of Contents{}

{{1~~}What is Python?}

{{2~~}Why Python?}

{{2.1~~}Easyness}

{{2.1.1~~}Example: Date-Time Program}

{{2.2~~}Flexibility}

{{2.3~~}Popularity}

{{2.4~~}Community}

{{3~~}Python Getting Started}

{{3.1~~}Python 2.7 vs Python 3.6}

{{4~~}Python Installation.}

{{5~~}Python Simple Calculator.}

{{6~~}Python text editors and IDEs.}

{{7~~}References and Useful Resources.}

    \subsection{What is Python?}\label{what-is-python}

    Python is an interpreted, high-level, programming language. It contains
many high level built-in functions and data structures. Its name comes
from \textbf{Monty Python} British comedy group. Python emphasizes on
code readability and rapid development. There are many features in the
language elaborating this characteristics such as \textbf{dynamic
typing, indentation, useful keywords naming}. These terms will be clear
in the upcoming sections of this document.

    \subsection{Why Python?}\label{why-python}

    There are many competitive programming languages that are leading the
commercial market nowadays. What is so special about python such that I
will , as a engineer, developer or even a user, choose python for my
next project? Well, this is a fair enough question to ask. The following
points will show Python potential and proficiency.

    \subsubsection{Easyness}\label{easyness}

    Python, as a rumor, was intended to target kids to make the concept of
programming easy for them. However, it scales up such that Nasa is using
it in its projects! Python syntax is interestingly easy to learn and
understand. The code is, probably, the most readable among all its
counterparts. The following example elaborates that. It is a program to
show the current day, time.

    \paragraph{Example: Date-Time Program}\label{example-date-time-program}

    \begin{Verbatim}[commandchars=\\\{\}]
{\color{incolor}In [{\color{incolor}32}]:} \PY{k+kn}{from} \PY{n+nn}{datetime} \PY{k}{import} \PY{n}{datetime}
         
         \PY{n}{now} \PY{o}{=} \PY{n}{datetime}\PY{o}{.}\PY{n}{now}\PY{p}{(}\PY{p}{)}
         
         \PY{n}{mm} \PY{o}{=} \PY{n+nb}{str}\PY{p}{(}\PY{n}{now}\PY{o}{.}\PY{n}{month}\PY{p}{)}
         
         \PY{n}{dd} \PY{o}{=} \PY{n+nb}{str}\PY{p}{(}\PY{n}{now}\PY{o}{.}\PY{n}{day}\PY{p}{)}
         
         \PY{n}{yyyy} \PY{o}{=} \PY{n+nb}{str}\PY{p}{(}\PY{n}{now}\PY{o}{.}\PY{n}{year}\PY{p}{)}
         
         \PY{n}{hour} \PY{o}{=} \PY{n+nb}{str}\PY{p}{(}\PY{n}{now}\PY{o}{.}\PY{n}{hour}\PY{p}{)}
         
         \PY{n}{mi} \PY{o}{=} \PY{n+nb}{str}\PY{p}{(}\PY{n}{now}\PY{o}{.}\PY{n}{minute}\PY{p}{)}
         
         \PY{n}{ss} \PY{o}{=} \PY{n+nb}{str}\PY{p}{(}\PY{n}{now}\PY{o}{.}\PY{n}{second}\PY{p}{)}
         
         \PY{n+nb}{print}\PY{p}{(}\PY{l+s+s1}{\PYZsq{}}\PY{l+s+s1}{date of today is: }\PY{l+s+s1}{\PYZsq{}}\PY{p}{)}
         \PY{n+nb}{print} \PY{p}{(}\PY{n}{mm} \PY{o}{+} \PY{l+s+s2}{\PYZdq{}}\PY{l+s+s2}{/}\PY{l+s+s2}{\PYZdq{}} \PY{o}{+} \PY{n}{dd} \PY{o}{+} \PY{l+s+s2}{\PYZdq{}}\PY{l+s+s2}{/}\PY{l+s+s2}{\PYZdq{}} \PY{o}{+} \PY{n}{yyyy} 
                \PY{o}{+} \PY{l+s+s2}{\PYZdq{}}\PY{l+s+s2}{ }\PY{l+s+s2}{\PYZdq{}} \PY{o}{+} \PY{n}{hour} \PY{o}{+} \PY{l+s+s2}{\PYZdq{}}\PY{l+s+s2}{:}\PY{l+s+s2}{\PYZdq{}} \PY{o}{+} \PY{n}{mi} \PY{o}{+} \PY{l+s+s2}{\PYZdq{}}\PY{l+s+s2}{:}\PY{l+s+s2}{\PYZdq{}} \PY{o}{+} \PY{n}{ss}\PY{p}{)}
\end{Verbatim}


    \begin{Verbatim}[commandchars=\\\{\}]
date of today is: 
3/18/2018 22:55:22

    \end{Verbatim}

    \subsubsection{Flexibility}\label{flexibility}

    \begin{itemize}
\tightlist
\item
  \textbf{Python Characteristics:} Python scripts are dynamically typed.
  Meaning that, the programmer needs not to explicitly define the type
  of the variable to be used. This is a great increase in readability
  and decrease of time cost in programming. Adding to that, Python
  provides the programmer many methods to solve a particular problem
  increasing programmers to develop programming flavors. Moreover,
  Python supports Object-Oriented Programming "OOP". This is a modern
  concept applied in the leading programming languages such as C++, C\#,
  Java and, of course, Python. It is used to modularize and encapsulate
  data types and their specific operations. It will be extrapolated
  later in this tutorial documents.
\end{itemize}

    \begin{itemize}
\tightlist
\item
  \textbf{Portability:} Code written in C will produce .exe file when
  compiled. That file will execute only on a windows machines. Same thin
  is happening for mac machines. Nevertheless, Java breaks this rule by
  its smart idea: Java Virtual Machine. Using this feature, you can
  write Java program and run it on Windows or Mac machine using this
  feature. Python follows the same rule. Python scripts can be running
  on any platform.
\end{itemize}

    \begin{itemize}
\tightlist
\item
  \textbf{Python frameworks:} Python frameworks are python scripts and
  modules helping the programmer to develop their applications on many
  platforms. The following points are just examples of \emph{what you
  can do with python}:

  \begin{itemize}
  \tightlist
  \item
    Web Development There are many libraries and frameworks letting you
    to develop web applications using your favourite programming
    language: \emph{Python}! Such frameworks are Django, Flusk,
    Pyramids.
  \item
    Data Analysis Data analysis is a giant part of python frameworks and
    libraries. Day to day, data is becoming the new gold of this
    century. Python is a leading software in this field. Many so popular
    and robust libraries are written in Python such as Numpy, Pandas,
    Sciypy and the library of Data Visualization Matplotlib. We may
    going a little bit deeper in this tutorial as we are going to study
    Python as a good alternative for Matlab.
  \item
    Machine Learning Machine learning is the formal name given to the
    field of extracting useful information out of the data. This field
    is well-known for its huge computation and time-machinery
    expensiveness. It also requires the programming to know, in-deep,
    many mathematical concepts such as probability, calculus and linear
    algebra. Python has many frame works utilizing this work for the
    programmer. The programmer needs not to be fully knowledgeable about
    such complicated stuff. These frameworks are NLTK for natural
    language processing, Tensorflow for building Artificial Neural
    Networks, and Scikit-learn for machine learning algorithms.
  \item
    Computer Vision. OpenCV with Python. Computer Vision, in simple
    words, interested in image manipulation, detection and recognition.
    All these fancy words are given this neat name, Image Processing.
    Many interesting areas are falling inside this field such as face
    detection and face recognition. These are part of Computer Vision
    which connects all these components altogether and they all falls
    under the machine learning and AI umbrella. Python has nice module
    implementations regarding this topics. OpenCV is an excellent
    example.
  \item
    GUI Development. Python, just like any other programming language,
    provides libraries to write GUI applications. GUI stands for
    Graphical User Interface which is the window you see when you start
    any desktop application. pyqt, Tkinter and Jython are examples of
    such libraries. There is also a \_ Game Development \_ python
    library known as pygame that can run on Android system as well.
  \end{itemize}
\end{itemize}

    \subsubsection{Popularity}\label{popularity}

    Python is widely used in many applications that are maintained and
developed using the giant computing and software companies in the world
such as Google, Facebook, Amazon etc. Youtube, Google, Quora, Instagram,
Spotify, Reddit, Pinterest, and Prezi websites are completely or
partially powered by Python. Python is the 4th Most-Used Language at
Github "Github is a coding projects websites where programmers use to
save their codes online". If Python was useless or was not a
cutting-edge technology providing integrated features, these large
companies will no rely on it to build their websites.

    \subsubsection{Community}\label{community}

    Python has a large community over the world. There is an annual
conference for python developers called
\href{https://www.pycon.org/}{PyCon}. The link:(https://www.pycon.org/).

Python also has a great community support at stackoverflow.com. The
website for programmers. It is the 5th largest community on the site
according to{[}6{]}. You will not be misled. Moreover, any question you
post on this site about python, it is the 3rd most likely to be answered
when compared to other popular programming languages.

    \subsection{Python Getting Started}\label{python-getting-started}

    \subsubsection{Python 2.7 vs Python 3.6}\label{python-2.7-vs-python-3.6}

    Python 2.7 and Python 3.6 are two main versions of Python. Python start
a new software with different criteras after Python 2.7 version. There
are numerous differences between the two versions summarized in {[}7{]}.
We will be exposed to these differences when appropriate. For this
tutorial, we will be using Python 3.6.

    Starting with the first difference, Python \texttt{print} statement is
changed from 2.7 to 3.x. in python 2.7, It was like the following:

\texttt{print\ \textquotesingle{}hello\ world\textquotesingle{}}

However, in python 3.6 it changes to:

\texttt{print(\textquotesingle{}hello\ world\textquotesingle{})}

    \begin{Verbatim}[commandchars=\\\{\}]
{\color{incolor}In [{\color{incolor}33}]:} \PY{n+nb}{print}\PY{p}{(}\PY{l+s+s1}{\PYZsq{}}\PY{l+s+s1}{hello world}\PY{l+s+s1}{\PYZsq{}}\PY{p}{)}
\end{Verbatim}


    \begin{Verbatim}[commandchars=\\\{\}]
hello world

    \end{Verbatim}

    \subsection{Python Installation.}\label{python-installation.}

    Python can be installed from their official website:
https://www.python.org/downloads/. There will be two illuminating
buttons at the header of the page for the two versions of python. You
may choose Python 3.6 to download.

After clicking this button, a compressed file will start downloading.
Once it completes downloading, you can install it as usual.

    \subsection{Python Simple Calculator.}\label{python-simple-calculator.}

    To run python, open the command line software on you OS and type python.
Now, you can write python scripts on the command line. I know that this
is an ugly user interface. Nevertheless, we will have more fancy choices
later. Try the following simple command to test language functionality:

    \begin{Verbatim}[commandchars=\\\{\}]
{\color{incolor}In [{\color{incolor}34}]:} \PY{n+nb}{print} \PY{p}{(}\PY{l+s+s1}{\PYZsq{}}\PY{l+s+s1}{Hello world}\PY{l+s+s1}{\PYZsq{}}\PY{p}{)}
\end{Verbatim}


    \begin{Verbatim}[commandchars=\\\{\}]
Hello world

    \end{Verbatim}

    Here, we are going to perform simple calculations. These calculations
may let you consider python as an alternative calculator of your OS one.
Try it, It is amazing!

    \textbf{Tips}:

\begin{itemize}
\tightlist
\item
  Python provides two methods of writing comments:
\end{itemize}

\texttt{\#\ This\ is\ a\ line\ comment}

\texttt{\textquotesingle{}\textquotesingle{}\textquotesingle{}\ This\ is\ block\ comment\textquotesingle{}\textquotesingle{}\textquotesingle{}}

\begin{itemize}
\tightlist
\item
  Unlike many popular programming languages, Python uses indentation
  rather than semi-colon. This will be explained later.
\end{itemize}

    The following operators are used: - \texttt{*} for multiplication. -
\texttt{+} for addition. - \texttt{-} for subtraction. - \texttt{/} for
division. - \texttt{//} for integer division. - \texttt{**} for power.

Note that in \emph{Python 2.7}, \texttt{/} and \texttt{//} are both used
for integer division. To save, the decimal part while performing
division, one needs to put .0 on one of the operands like
\texttt{11/2.0}.

    \begin{Verbatim}[commandchars=\\\{\}]
{\color{incolor}In [{\color{incolor}35}]:} \PY{c+c1}{\PYZsh{} this is a line comment}
         \PY{l+s+sd}{\PYZsq{}\PYZsq{}\PYZsq{}}
         \PY{l+s+sd}{This is a block comment!}
         \PY{l+s+sd}{\PYZsq{}\PYZsq{}\PYZsq{}}
\end{Verbatim}


\begin{Verbatim}[commandchars=\\\{\}]
{\color{outcolor}Out[{\color{outcolor}35}]:} '\textbackslash{}nThis is a block comment!\textbackslash{}n'
\end{Verbatim}
            
    \begin{Verbatim}[commandchars=\\\{\}]
{\color{incolor}In [{\color{incolor}36}]:} \PY{c+c1}{\PYZsh{} Some Calculations:}
\end{Verbatim}


    \begin{Verbatim}[commandchars=\\\{\}]
{\color{incolor}In [{\color{incolor}37}]:} \PY{l+m+mi}{10}\PY{o}{\PYZhy{}}\PY{l+m+mi}{5}
\end{Verbatim}


\begin{Verbatim}[commandchars=\\\{\}]
{\color{outcolor}Out[{\color{outcolor}37}]:} 5
\end{Verbatim}
            
    \begin{Verbatim}[commandchars=\\\{\}]
{\color{incolor}In [{\color{incolor}38}]:} \PY{l+m+mi}{10}\PY{o}{+}\PY{l+m+mi}{5}
\end{Verbatim}


\begin{Verbatim}[commandchars=\\\{\}]
{\color{outcolor}Out[{\color{outcolor}38}]:} 15
\end{Verbatim}
            
    \begin{Verbatim}[commandchars=\\\{\}]
{\color{incolor}In [{\color{incolor}39}]:} \PY{l+m+mi}{5}\PY{o}{\PYZhy{}}\PY{l+m+mi}{10}
\end{Verbatim}


\begin{Verbatim}[commandchars=\\\{\}]
{\color{outcolor}Out[{\color{outcolor}39}]:} -5
\end{Verbatim}
            
    \begin{Verbatim}[commandchars=\\\{\}]
{\color{incolor}In [{\color{incolor}40}]:} \PY{l+m+mi}{10}\PY{o}{*}\PY{l+m+mi}{5}
\end{Verbatim}


\begin{Verbatim}[commandchars=\\\{\}]
{\color{outcolor}Out[{\color{outcolor}40}]:} 50
\end{Verbatim}
            
    \begin{Verbatim}[commandchars=\\\{\}]
{\color{incolor}In [{\color{incolor}41}]:} \PY{l+s+sd}{\PYZsq{}\PYZsq{}\PYZsq{}}
         \PY{l+s+sd}{notice that / returns a real value }
         \PY{l+s+sd}{while // returns an integer value where the decimal part is truncrated.}
         \PY{l+s+sd}{\PYZsq{}\PYZsq{}\PYZsq{}}
         \PY{l+m+mi}{11}\PY{o}{/}\PY{l+m+mi}{5}
\end{Verbatim}


\begin{Verbatim}[commandchars=\\\{\}]
{\color{outcolor}Out[{\color{outcolor}41}]:} 2.2
\end{Verbatim}
            
    \begin{Verbatim}[commandchars=\\\{\}]
{\color{incolor}In [{\color{incolor}42}]:} \PY{l+m+mi}{11}\PY{o}{/}\PY{o}{/}\PY{l+m+mi}{5}
\end{Verbatim}


\begin{Verbatim}[commandchars=\\\{\}]
{\color{outcolor}Out[{\color{outcolor}42}]:} 2
\end{Verbatim}
            
    \begin{Verbatim}[commandchars=\\\{\}]
{\color{incolor}In [{\color{incolor}43}]:} \PY{l+m+mi}{2}\PY{o}{*}\PY{o}{*}\PY{l+m+mi}{3}
\end{Verbatim}


\begin{Verbatim}[commandchars=\\\{\}]
{\color{outcolor}Out[{\color{outcolor}43}]:} 8
\end{Verbatim}
            
    \begin{Verbatim}[commandchars=\\\{\}]
{\color{incolor}In [{\color{incolor}44}]:} \PY{l+m+mi}{2}\PY{o}{*}\PY{o}{*}\PY{o}{\PYZhy{}}\PY{l+m+mi}{3}
\end{Verbatim}


\begin{Verbatim}[commandchars=\\\{\}]
{\color{outcolor}Out[{\color{outcolor}44}]:} 0.125
\end{Verbatim}
            
    \begin{Verbatim}[commandchars=\\\{\}]
{\color{incolor}In [{\color{incolor}45}]:} \PY{l+m+mi}{4}\PY{o}{*}\PY{o}{*}\PY{l+m+mf}{0.5}
\end{Verbatim}


\begin{Verbatim}[commandchars=\\\{\}]
{\color{outcolor}Out[{\color{outcolor}45}]:} 2.0
\end{Verbatim}
            
    \subsection{Python text editors and
IDEs.}\label{python-text-editors-and-ides.}

    As we agreed above, we will present in this section some text editors
that are definitely best substitutions of the ugly command line
interface.

The list goes below:

    \begin{itemize}
\tightlist
\item
  \textbf{Visual Studio Code.}
\item
  \textbf{Atom.}
\item
  \textbf{Notepad++.}
\item
  \textbf{Sublime text.}
\item
  \textbf{Emacs.}
\item
  \textbf{Wing IDE.}
\item
  \textbf{PyCharm IDE.}
\end{itemize}

    Moreover, we can use \textbf{Jupyter Notebook} for writing documents
with code that can be exported later to a \(LaTeX\) document. This
technology is available under \emph{Anaconda} distribution and will be
discussed later.

    \subsection{References and Useful
Resources.}\label{references-and-useful-resources.}

    1- \href{https://www.python.org/doc/essays/blurb/}{What is python?}.
Python official website: https://www.python.org/doc/essays/blurb/

2-
\href{http://www.pythonforbeginners.com/code-snippets-source-code/python-code-examples}{Python
code examples}, The website is:
http://www.pythonforbeginners.com/code-snippets-source-code/python-code-examples

3- \href{https://www.youtube.com/watch?v=hxGB7LU4i1I}{What can you do
with python}. The link to this Youtube videw is:
https://www.youtube.com/watch?v=hxGB7LU4i1I.

4- \href{https://www.python.org/about/apps/}{Applications for Python}.
The link is: https://www.python.org/about/apps/. It is from the official
python website.

5-
\href{https://www.shuup.com/blog/25-of-the-most-popular-python-and-django-websites/}{Top
25 most popular websites built with python} The link to this blog is:
https://www.shuup.com/blog/25-of-the-most-popular-python-and-django-websites/.

6- \href{http://www.bestprogramminglanguagefor.me/why-learn-python}{why
to learn python} link:
http://www.bestprogramminglanguagefor.me/why-learn-python.

7-
\href{http://sebastianraschka.com/Articles/2014_python_2_3_key_diff.html}{key
differences between Python2.7 and Python 3.6}. The link to the blog is:
http://sebastianraschka.com/Articles/2014\_python\_2\_3\_key\_diff.html


    % Add a bibliography block to the postdoc
    
    
    
    \end{document}
